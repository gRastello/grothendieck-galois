\documentclass[italian, 12pt, reqno]{article}
\usepackage[english]{babel}
\usepackage[T1]{fontenc}

% Stylistic points.
\usepackage{geometry}
\geometry{
  a4paper,
  right=2.5cm,
  left=2.5cm,
  top=2.5cm,
  bottom=2.5cm,
  %showframe,
}

\usepackage{mathpazo} % Math Palatino font.

% Links and references.
\usepackage{xcolor}
\definecolor{Myblue}{rgb}{0,0,0.6}
\usepackage[a4paper,colorlinks,citecolor=Myblue,linkcolor=Myblue,urlcolor=Myblue,pdfpagemode=None]{hyperref}

% Necessities for math.
\usepackage{amsmath, amscd, amssymb, mathrsfs, accents, amsfonts, amsthm}

\newtheoremstyle{myteo}{\topsep}{\topsep}
	{}
	{}
	{\bfseries}
        {.}
	{2pt}
	{\thmname{#1}\thmnumber{ #2}\thmnote{ (#3)}}
\theoremstyle{myteo}

\newtheorem{theorem}{Theorem}[section]
\newtheorem{proposition}[theorem]{Proposition}
\newtheorem{lemma}[theorem]{Lemma}
\newtheorem{corollary}[theorem]{Corollary}
\newtheorem{definition}[theorem]{Definition}
\newtheorem{example}[theorem]{Example}
\newtheorem{remark}[theorem]{Remark}
\newtheorem{notation}[theorem]{Notation}

\numberwithin{equation}{section}

\usepackage{tikz}
\usetikzlibrary{cd}
\usetikzlibrary{fadings}

% Figures stuff.
\usepackage{caption}
\renewcommand{\thefigure}{\arabic{section}.\arabic{figure}}

% Lists stuff.
\usepackage{enumitem}
\setenumerate{label=(\arabic*)}

% Commands.
\newcommand{\cat}[1]{\mathscr{#1}}
\newcommand{\dual}[1]{#1^{\text{op}}}
\newcommand{\aut}{\text{Aut}}
\newcommand{\gset}{\text{\textbf{GSet}}}
\newcommand{\fix}[1]{\text{Fix}(#1)}
\newcommand{\fun}[3]{#1\colon #2\to #3}

% Operators
\DeclareMathOperator{\im}{im}

\begin{document}
\title{Grothendieck's Galois Theory}
\author{Gabriele Rastello}
\maketitle

\section{Section 1}
\label{sec:1}

\begin{definition}
  \label{def:strit_epi}
  In a category \(\cat{C}\) an arrow \(\fun{f}{X}{Y}\) is a \textbf{strict epimorphism} if it is the joint coequalizer of all the arrows it coequalizes.
  This means that any arrow \(\fun{g}{X}{Z}\) such that \(g\circ x = g\circ y\) for any \(\fun{x,y}{C}{X}\) such that \(f\circ x = f\circ y\) there exists a unique arrow \(\fun{h}{Y}{Z}\) such that \(h\circ f = g\).
  Refer to Figure \ref{diagram:strict_epi}.
\end{definition}

\begin{figure}[h]
  \begin{center}
    \begin{tikzcd}[sep=huge]
      C \ar[r, shift right=1, "x"'] \ar[r, shift left=1, "y"] & X \ar[r, "f"] \ar[dr, "g"] & Y \ar[d, dashed, "h"]\\
      & & Z
    \end{tikzcd}
  \end{center}
  \caption{}
  \label{diagram:strict_epi}
\end{figure}

\begin{remark}
  \label{rem:strict_epi}
  Strict epimorphisms are coequalizers, thus epimorphisms (as the name implies).
\end{remark}

\begin{remark}
  \label{rem:strict_epi_plus_mono}
  If an arrow is both a stric epimorphism and a monomorphism then it is an epimorphism.
\end{remark}

\begin{definition}
  \label{def:action}
  Let \(H\) be a group, \(A\) an object of \(\cat{C}\) and \(G = \aut(A)\) the group of automorphisms of \(A\) in \(\cat{C}\) i.e. the group whose underlying set is the set of isomorphisms of type \(A\to A\) of \(\cat{C}\) and whose operation is composition in \(\cat{C}\).
  An \textbf{action} of \(H\) on \(A\) is a group homomorphism \(H \to G\).
\end{definition}

\begin{notation}
  \label{not:action}
  Given an action of a group \(H\) on an object \(A\) of \(\cat{C}\) we denote, with a slight abuse of notation, the automorphism of \(A\) associated to \(h\in H\) by the same symbol \(h\).
\end{notation}

\begin{definition}
  \label{def:quotient}
  If \(H\) acts on \(A\) as defined in \ref{def:action} we define the quotient of \(A\) by \(H\) in \(\cat{C}\) to be an element \(A/H\) of \(\cat{C}\) equipped with an arrow \(\fun{q}{A}{A/H}\) such that:
  \begin{enumerate}
  \item for all \(h\in H\) \(q\circ h = q\) holds,
  \item for any \(\fun{x}{A}{X}\) such that \(x\circ h = x\) for all \(h\in H\) there exists a unique arrow \(\fun{\varphi}{A/H}{X}\) such that \(x = \varphi\circ q\).
  \end{enumerate}
  See also Figure \ref{diagram:quotient}.
\end{definition}

\begin{remark}
  \label{rem:uniqueness_of_quotient}
  Quotients are defined by a universal property, thus are unique up to unique isomorphism and we can speak of ``the'' quotient of \(A\) by \(H\) instead of ``a'' quotient of \(A\) by \(H\).
\end{remark}

\begin{notation}
  \label{not:quotient}
  Sometimes we use the sentence ``the quotient of \(A\) by \(H\)'' to refer to the object \(A/H\), some others to the arrow \(\fun{q}{A}{A/H}\); the context should be enough to differentiate between the two cases.
\end{notation}

\begin{figure}[h]
  \begin{center}
    \begin{tikzcd}[sep=huge]
      A \ar[loop left, "h"] \ar[r, "q"] \ar[rd, "x"] & A/H \ar[d, "\varphi"]\\
      & X
    \end{tikzcd}
  \end{center}
  \caption{}
  \label{diagram:quotient}
\end{figure}

\begin{remark}
  \label{rem:quotient_are_strict_epi}
  Consider a quotient \(\fun{q}{A}{A/H}\); by condition (1) above \(q\circ h = q = q\circ 1_A\) so \(q\) coequalizes all the pairs \((h, 1_A)\), for \(h\in H\).
  If another arrow \(\fun{x}{A}{X}\) coequalizes all the pairs that \(q\) does then this arrow is such that \(x\circ h = x\circ 1_A = x\) for all \(h\in H\) and thus, by condition (2), we have a unique factorization \(x = \varphi\circ q\).
  This proves that all quotients are strict epimorphisms.
\end{remark}

\begin{remark}
  \label{rem:orbits}
  Let \(G\) be a group, \(\gset\) the category of \(G\)-sets and \(G\)-invariant maps and \(A\) an object of \(\gset\).
  In this category Definition \ref{def:quotient} yelds the familiar notion of the set of all orbits of an action: \(A/G\) is the set of orbits of \(A\).
\end{remark}

\begin{remark}
  \label{rem:transitive_actions}
  Consider again \(\gset\).
  The underlying set of \(G\) (that we also denote as \(G\)) is a G-set with the action given by left multiplication in \(G\); we call this the \textbf{canonical action} of \(G\) on itself.
  Let \(\fun{\varphi}{G}{E}\) be a G-invariant map; it is easy to see that such a \(\varphi\), by virtue of being G-invariant, is determined uniquely by the value \(\varphi(e)\), where \(e\) is the neutral element of \(G\).

  Let now \(E\) be a transitive G-set i.e. a set upon which the action of \(G\) is transitive i.e. such that \(E/G = \{*\}\).
  Fix an \(x\in E\) and let \(\varphi_x\) be the G-invarian map defined by \(\varphi_x(e) = x\); we argue that \(\fun{\varphi_x}{G}{E}\) makes \(E\) into a quotient of \(G\) by the subgroup
  \[H = \fix{x} = \{g\in G\colon gx = x\}.\]
  Indeed by using the definition of \(H\) and the fact that \(\varphi\) is G-invariant we have
  \[(\varphi\circ h)(e) = \varphi(h(e)) = h(\varphi(e)) = h(x) = x.\]
  for all \(h\in H\).
  Moreover let \(\fun{g}{G}{F}\) satisfy (1) of Definition \ref{def:quotient}; as we discussed above \(g\) is entirely determined by the image of \(e\) so we obtain (2) defining an arrow \(\fun{f}{E}{F}\) by \(f(x) = g(e)\).
  The situation is depicted in Figure \ref{diagram:transitive_actions}.

  Trivially \(G\) is a transitive G-set and for any \(g\in G\) \(G/\fix{g}\) is transitive as well so we have that an object \(E\in\gset\) is transitive if and only if it is isomorphic to \(G/H\) where \(H = \fix{x}\) for (any) \(x\in E\).
\end{remark}

\begin{figure}
  \begin{center}
    \begin{tikzcd}[sep=huge]
      G \ar[loop left, "h"] \ar[dr, "g"] \ar[r, "\varphi"] & E \cong G/H \ar[d, dashed, "f"]\\
      & F
    \end{tikzcd}
  \end{center}
  \caption{}
  \label{diagram:transitive_actions}
\end{figure}

\end{document}

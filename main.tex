\documentclass[italian, 12pt, reqno]{article}
\usepackage[english]{babel}
\usepackage[T1]{fontenc}

% Stylistic points.
\usepackage{geometry}
\geometry{
  a4paper,
  right=2.5cm,
  left=2.5cm,
  top=2.5cm,
  bottom=2.5cm,
  %showframe,
}

\usepackage{mathpazo} % Math Palatino font.

% Links and references.
\usepackage{xcolor}
\definecolor{Myblue}{rgb}{0,0,0.6}
\usepackage[a4paper,colorlinks,citecolor=Myblue,linkcolor=Myblue,urlcolor=Myblue,pdfpagemode=None]{hyperref}

% Necessities for math.
\usepackage{amsmath, amscd, amssymb, mathrsfs, accents, amsfonts, amsthm}

\newtheoremstyle{myteo}{\topsep}{\topsep}
	{}
	{}
	{\bfseries}
        {.}
	{2pt}
	{\thmname{#1}\thmnumber{ #2}\thmnote{ (#3)}}
\theoremstyle{myteo}

\newtheorem{theorem}{Theorem}[section]
\newtheorem{proposition}[theorem]{Proposition}
\newtheorem{lemma}[theorem]{Lemma}
\newtheorem{corollary}[theorem]{Corollary}
\newtheorem{definition}[theorem]{Definition}
\newtheorem{example}[theorem]{Example}
\newtheorem{remark}[theorem]{Remark}
\newtheorem{notation}[theorem]{Notation}

\numberwithin{equation}{section}

\usepackage{tikz}
\usetikzlibrary{cd}
\usetikzlibrary{fadings}

% Figures stuff.
\usepackage{caption}
\renewcommand{\thefigure}{\arabic{section}.\arabic{figure}}

% Lists stuff.
\usepackage{enumitem}
\setenumerate{label=(\arabic*)}

% Commands.
\newcommand{\cat}[1]{\mathscr{#1}}
\newcommand{\dual}[1]{#1^{\text{op}}}
\newcommand{\aut}{\text{Aut}}
\newcommand{\gset}{\text{\textbf{GSet}}}
\newcommand{\tgset}{\text{\textbf{GSet}}^t}
\newcommand{\fix}[1]{\text{Fix}(#1)}
\newcommand{\homs}[2]{[#1, #2]}
\newcommand{\fun}[3]{#1\colon#2\to #3}
\newcommand{\nonamefundef}[4]{\begin{align*}
                             \begin{tikzcd}[sep=large]#1\end{tikzcd} &\longrightarrow \begin{tikzcd}[sep=large]#2\end{tikzcd}\\
                             \begin{tikzcd}[sep=large]#3\end{tikzcd} &\longmapsto \begin{tikzcd}[sep=large]#4\end{tikzcd}
                           \end{align*}}
\newcommand{\fundef}[5]{\begin{align*}
                             #1\colon\begin{tikzcd}[sep=large]#2\end{tikzcd} &\longrightarrow \begin{tikzcd}[sep=large]#3\end{tikzcd}\\
                             \begin{tikzcd}[sep=large]#4\end{tikzcd} &\longmapsto \begin{tikzcd}[sep=large]#5\end{tikzcd}
\end{align*}}

% Operators
\DeclareMathOperator{\im}{im}

\begin{document}
\title{Grothendieck's Galois Theory}
\author{Gabriele Rastello}
\maketitle

\section{Group actions and adjoints}
\label{sec:1}

\begin{definition}
  \label{def:strit_epi}
  In a category \(\cat{C}\) an arrow \(\fun{f}{X}{Y}\) is a \textbf{strict epimorphism} if it is the joint coequalizer of all the arrows it coequalizes.
  This means that any arrow \(\fun{g}{X}{Z}\) such that \(g\circ x = g\circ y\) for any \(\fun{x,y}{C}{X}\) such that \(f\circ x = f\circ y\) there exists a unique arrow \(\fun{h}{Y}{Z}\) such that \(h\circ f = g\).
  Refer to Figure \ref{diagram:strict_epi}.
\end{definition}

\begin{figure}[h]
  \begin{center}
    \begin{tikzcd}[sep=huge]
      C \ar[r, shift right=1, "x"'] \ar[r, shift left=1, "y"] & X \ar[r, "f"] \ar[dr, "g"] & Y \ar[d, dashed, "h"]\\
      & & Z
    \end{tikzcd}
  \end{center}
  \caption{}
  \label{diagram:strict_epi}
\end{figure}

\begin{remark}
  \label{rem:strict_epi}
  Strict epimorphisms are coequalizers, thus epimorphisms (as the name implies).
\end{remark}

\begin{remark}
  \label{rem:strict_epi_plus_mono}
  If an arrow is both a stric epimorphism and a monomorphism then it is an epimorphism.
\end{remark}

\begin{definition}
  \label{def:action}
  Let \(H\) be a group, \(A\) an object of \(\cat{C}\) and \(G = \aut(A)\) the group of automorphisms of \(A\) in \(\cat{C}\) i.e. the group whose underlying set is the set of isomorphisms of type \(A\to A\) of \(\cat{C}\) and whose operation is composition in \(\cat{C}\).
  An \textbf{action} of \(H\) on \(A\) is a group homomorphism \(H \to G\).
\end{definition}

\begin{notation}
  \label{not:action}
  Given an action of a group \(H\) on an object \(A\) of \(\cat{C}\) we denote, with a slight abuse of notation, the automorphism of \(A\) associated to \(h\in H\) by the same symbol \(h\).
\end{notation}

\begin{definition}
  \label{def:quotient}
  If \(H\) acts on \(A\) as defined in \ref{def:action} we define the quotient of \(A\) by \(H\) in \(\cat{C}\) to be an element \(A/H\) of \(\cat{C}\) equipped with an arrow \(\fun{q}{A}{A/H}\) such that:
  \begin{enumerate}
  \item for all \(h\in H\) \(q\circ h = q\) holds,
  \item for any \(\fun{x}{A}{X}\) such that \(x\circ h = x\) for all \(h\in H\) there exists a unique arrow \(\fun{\varphi}{A/H}{X}\) such that \(x = \varphi\circ q\).
  \end{enumerate}
  See also Figure \ref{diagram:quotient}.
\end{definition}

\begin{remark}
  \label{rem:uniqueness_of_quotient}
  Quotients are defined by a universal property, thus are unique up to unique isomorphism and we can speak of ``the'' quotient of \(A\) by \(H\) instead of ``a'' quotient of \(A\) by \(H\).
\end{remark}

\begin{notation}
  \label{not:quotient}
  Sometimes we use the sentence ``the quotient of \(A\) by \(H\)'' to refer to the object \(A/H\), some others to the arrow \(\fun{q}{A}{A/H}\); the context should be enough to differentiate between the two cases.
\end{notation}

\begin{figure}[h]
  \begin{center}
    \begin{tikzcd}[sep=huge]
      A \ar[loop left, "h"] \ar[r, "q"] \ar[rd, "x"] & A/H \ar[d, "\varphi"]\\
      & X
    \end{tikzcd}
  \end{center}
  \caption{}
  \label{diagram:quotient}
\end{figure}

\begin{remark}
  \label{rem:quotient_are_strict_epi}
  Consider a quotient \(\fun{q}{A}{A/H}\); by condition (1) above \(q\circ h = q = q\circ 1_A\) so \(q\) coequalizes all the pairs \((h, 1_A)\), for \(h\in H\).
  If another arrow \(\fun{x}{A}{X}\) coequalizes all the pairs that \(q\) does then this arrow is such that \(x\circ h = x\circ 1_A = x\) for all \(h\in H\) and thus, by condition (2), we have a unique factorization \(x = \varphi\circ q\).
  This proves that all quotients are strict epimorphisms.
\end{remark}

\begin{remark}
  \label{rem:orbits}
  Let \(G\) be a group, \(\gset\) the category of \(G\)-sets and \(G\)-invariant maps and \(A\) an object of \(\gset\).
  In this category Definition \ref{def:quotient} yelds the familiar notion of the set of all orbits of an action: \(A/G\) is the set of orbits of \(A\).
\end{remark}

\begin{remark}
  \label{rem:transitive_actions}
  Consider again \(\gset\).
  The underlying set of \(G\) (that we also denote as \(G\)) is a G-set with the action given by left multiplication in \(G\); we call this the \textbf{canonical action} of \(G\) on itself.
  Let \(\fun{\varphi}{G}{E}\) be a G-invariant map; it is easy to see that such a \(\varphi\), by virtue of being G-invariant, is determined uniquely by the value \(\varphi(e)\), where \(e\) is the neutral element of \(G\).

  Let now \(E\) be a transitive G-set i.e. a set upon which the action of \(G\) is transitive i.e. such that \(E/G = \{*\}\).
  Fix an \(x\in E\) and let \(\varphi_x\) be the G-invarian map defined by \(\varphi_x(e) = x\); we argue that \(\fun{\varphi_x}{G}{E}\) makes \(E\) into a quotient of \(G\) by the subgroup
  \[H = \fix{x} = \{g\in G\colon gx = x\}.\]
  Indeed by using the definition of \(H\) and the fact that \(\varphi\) is G-invariant we have
  \[(\varphi\circ h)(e) = \varphi(h(e)) = h(\varphi(e)) = h(x) = x.\]
  for all \(h\in H\).
  Moreover let \(\fun{g}{G}{F}\) satisfy (1) of Definition \ref{def:quotient}; as we discussed above \(g\) is entirely determined by the image of \(e\) so we obtain (2) defining an arrow \(\fun{f}{E}{F}\) by \(f(x) = g(e)\).
  The situation is depicted in Figure \ref{diagram:transitive_actions}.

  Trivially \(G\) is a transitive G-set and for any \(g\in G\) \(G/\fix{g}\) is transitive as well so we have that an object \(E\in\gset\) is transitive if and only if it is isomorphic to \(G/H\) where \(H = \fix{x}\) for (any) \(x\in E\).
\end{remark}

\begin{figure}
  \begin{center}
    \begin{tikzcd}[sep=huge]
      G \ar[loop left, "h"] \ar[dr, "g"] \ar[r, "\varphi"] & E \cong G/H \ar[d, dashed, "f"]\\
      & F
    \end{tikzcd}
  \end{center}
  \caption{}
  \label{diagram:transitive_actions}
\end{figure}

For the rest of the section fix a category \(\cat{C}\), an object \(A\in\cat{C}\) and let \(G = \aut(A)\).

\begin{proposition}
  \label{prop:action_hom_set}
  Consider a subgroup \(H\leq G\) and an object \(X\in\cat{C}\). \(H\) acts on the hom-set \([A, X]\) as follows\footnote{Since an action as of Definition \ref{def:action} is a map that sends elements of a group to arrows it is, in this case, equivalent to give the definition of an action by uncurrying.}:
  \nonamefundef{H\times [A, X]}{[A, X]}{(h, x)}{h\cdot x = x \circ h.}
\end{proposition}

\begin{remark}
  \label{rem:hom_functor}
  Assume that the action \(G\times [A, X] \to [A, X]\) is transitive and let \(\tgset\) be the category of transitive \(G\)-sets (a subcategory of \(Gset\)).
  Then we have a functor
  \fundef{[A, -]_G}{\cat{C}}{\tgset}{X \ar[d, "f"] \\ Y}{\homs{A}{X}_G \ar[d, "f_*"] \\ \homs{A}{Y}_G}
  where we indicate with \(\homs{A}{X}_G\) the hom-set \([A, X]\) upon which \(G\) acts as described in Proposition \ref{prop:action_hom_set} and \(f_*\) is post-composition with \(f\).
  It is easy to check that \(f_*\) is indeed \(G\)-invariant.
\end{remark}

\begin{remark}
  \label{rem:adjoints}
  Consider an object \(E\in\tgset\), pick an element \(x_0\in E\) and let \(H = \fix{x_0}\leq G\) (the choice of \(x_0\) is irrelevant as \(E\) is transitive).
  Moreover assume that \(\cat{C}\) has quotients of \(A\) by any subgroup of \(G\).

  By what we observed in Remark \ref{rem:transitive_actions} we have a bijection between elements of \(E\) and arrows of type \(G\to E\).
  Consider then \(f\in\homs{A}{X}\) and its corresponding arrow \(\fun{\varphi}{G}{\homs{A}{X}_G}\); we claim that \(f\) factors through \(A/H\) if and only if \(\varphi\) factors through \(E \cong G/H\) (see Figure \ref{diagram:bijection}).
  Indeed \(f\) factors if and only if \(f\circ h = f\) for all \(h\in H\), by using the fact that \(\varphi\) is \(G\)-invariant we obtain
  \[\varphi(h(e)) = h(\varphi(e)) = h(f) = f\circ h = f\]
  and, since \(\varphi\) is uniquely determined by \(\varphi(e)\),  \(\varphi\circ h =\varphi\) for all \(h\in H\); this happens if and only if \(\varphi\) factors through \(E\cong G/H\).

  This gives us, for each \(X\in\cat{C}\) and \(E\in\tgset\), a bijection
  \[[A/H, X] \cong \tgset(E, [A, X]_G)\]
  natural in \(X\) in \(E\) (see Section \ref{sec:appendixes}).
  Thus, if \(\cat{C}\) has quotients of \(A\) by subgroups of \(G\) we have an adjunction \(A\times_G - \dashv \homs{A}{-}_G\) where \(A\times_GE = A/H\) for \(H = \fix{x_0}\) and \(x_0\in E\) as above\footnote{We use the notation \(A\times_G-\) for the left adjoint in ``honor'' of the tensor-hom adjunction.}.
\end{remark}

\begin{figure}[h]
  \begin{center}
    \begin{tikzcd}[sep=huge]
      A \ar[loop left, "h"] \ar[r, ""] \ar[dr, "f"] & A/H \ar[d, dashed, ""]\\
      & X
    \end{tikzcd}\hspace{2cm}
    \begin{tikzcd}[sep=huge]
      G \ar[loop left, "h"] \ar[r, ""] \ar[dr, "\varphi"] & E\cong G/H \ar[d, dashed, ""]\\
      & \homs{A}{X}_G
    \end{tikzcd}
  \end{center}
  \caption{}
  \label{diagram:bijection}
\end{figure}

Our main problem will be that of finding conditions on \(\cat{C}\) that make this adjunction ineto an equivalence of categories.

\section{Categorial axiomatization of Galois Theory}
\label{sec:axioms}

Through this section fix a category \(\cat{C}\) and an object \(A\in\cat{C}\).

\begin{definition}
  \label{def:axioms}
  We define the following axioms.
  \begin{enumerate}
  \item For every \(X\in\cat{C}\) there is at least a map of type \(A\to X\) and all maps \(A\to X\) are strict epimorphisms.
  \item For any subgroup \(H\leq \aut(A)\) the quotient \(\fun{q}{A}{A/H}\) exists and is preserved by \(\fun{\homs{A}{-}}{\cat{C}}{\gset}\).
  \item Every endomorphism of \(A\) is an isomorphism i.e. \(\homs{A}{A} = \aut(A)\).
  \end{enumerate}
\end{definition}

\begin{remark}
  \label{rem:all_strict_epi}
  It is known that if \(f\circ g\) is a strict epimorphism then so is \(g\).
  Thus it follows from Axiom (1) that every arrow \(X\to Y\) in \(\cat{C}\) is a strict epimorphism.
\end{remark}

\begin{proposition}
  \label{prop:hom_is_faithful}
  Axiom (1) implies that \(\homs{A}{-}\) is faithful, reflects monomorphisms and isomorphisms.
\end{proposition}

\begin{proof}
  Consider arrows \(\fun{f,g}{X}{Y}\in\cat{C}\) such that \(\homs{A}{f} = \homs{A}{g}\); that is \(f_* = g_*\).
  By Axiom (1) let \(\fun{h}{A}{X}\) be a third arrow of \(\cat{C}\) then we have \(f_*(h) = g_*(h)\) i.e. \(f\circ h = g\circ h\).
  Again by Axiom (1) \(h\) is an epimorphism (really, a strong one) and thus we obtain \(f = g\); that is: \(\homs{A}{-}\) is faithful.
  
  It is well known that every faithful functor reflects monomorphisms.
  Because of this if \(f\in\gset\) is an isomorphism and \(g\in\cat{C}\) is such that \(\homs{A}{g} = f\) then \(g\) is a monomorphism too; but, as an arrow of \(\cat{C}\), \(g\) is also a strict epimorphism and thus an isomorphism.
\end{proof}

\section{Appendixes}
\label{sec:appendixes}

\subsection{Naturality of \(\homs{A/H}{X} \cong \tgset(E, \homs{A}{X}_G)\)}
Fix a category \(\cat{C}\), an element \(A\in\cat{C}\) and let \(G = \aut(A)\).
Let's indicate with \(\psi_{EX}\) the bijection between \(\homs{A/H}{X}\) and \(\tgset(E, \homs{A}{X}_G)\) described in Remark \ref{rem:adjoints}; we shall prove that it is natural in both \(X\in\cat{C}\) and \(E\in\tgset\).

\begin{proof}[Naturality in \(X\)]
  \label{proof:naturality_in_X}
  Given an arrow \(\fun{f}{X}{Y}\) of \(\cat{C}\) we want to prove that \(\psi_{EX}\) are the components of a natural transformation \(\homs{A/H}{-}\Rightarrow \tgset(E, \homs{A}{-}_G)\) i.e that the following diagram commutes.
  \begin{center}
    \begin{tikzcd}[sep=huge]
      \homs{A/H}{X} \ar[r, "\psi_{EX}"] \ar[d, "f_*"] & \tgset(E, \homs{A}{X}_G) \ar[d, "(f_*)_*"] \\
      \homs{A/H}{Y} \ar[r, "\psi_{EY}"] & \tgset(E, \homs{A}{Y}_G)
    \end{tikzcd}
  \end{center}
  Recall that here \(H = \fix{x_0}\) for \(x_0\in E\).
  Pick any \(x\in \homs{A/H}{X}\) and let \(\fun{q}{A}{A/H}\) be the quotient arrow (the quotient exists because \(\cat{C}\) is assumed to have all quotients by subgroups of \(G\)); then chasing \(x\) through the diagram down the two possible ways yields two arrows in \(\tgset\) of type \(E\to\homs{A}{Y}_G\).
  Since \(E\) is transitive we have \(E \cong G/H\) and this means that arrows of type \(E\to\homs{A}{Y}_G\) are determined uniquely by the image of \(e\in E \cong G/H\) (refer to the right part of Diagram \ref{diagram:bijection}).
  Keeping this in mind the following computations shows that the square commutes.
  \[((f_*)_*\circ \psi_{EX})(x)(e) = (f_*\circ \psi_{EX}(x))(e) = f_*(\psi_{EX}(x)(e)) = f_*(x \circ q) = f\circ x\circ q,\]
  \[(\psi_{EY}\circ f_*)(x)(e) = \psi_{EX}(f\circ x)(e) = f\circ x\circ q.\]
\end{proof}

\begin{proof}[Naturality in \(E\)]
  \label{proof:naturality_in_E}
  We want to prove that \(\psi_{EX}\) are the components of a natural transformation \(\homs{A/-}{X}\Rightarrow\tgset(-, \homs{A}{X}_G)\), but before drawing the naturality square as above we shall note that this case is slightly complicated by the fact that it is not immediately clear how the functor \(\homs{A/-}{X}\) acts on arrows.
  Indeed let \(\fun{f}{E\cong G/H}{F\cong G/H'}\) be an arrow of \(\tgset\) with \(H = \fix{x_0}, H' = \fix{f(x_0)}\) and \(x_0\in E\); these choices are justified by the fact that both \(E\) and \(F\) are transitive \(G\)-sets.
  Now notice that, for \(h\in H\), \(f(h\cdot x_0) = f(x_0)\) by definition of \(H\) and \(f(h\cdot x_0) = h\cdot f(x_0)\) by \(G\)-invariance of \(f\); thus \(h\in H'\) and thus \(H\subset H'\).
  This means that if we let \(\fun{q}{A}{A/H}, \fun{q'}{A}{A/H'}\) be the quotients in \(\cat{C}\) then there is a unique arrow \(\fun{\widetilde{f}}{A/H}{A/H'}\) such that \(\widetilde{f}\circ q = q'\).
  The functor \(\homs{A/-}{X}\) then sends the arrow \(\fun{f}{E}{F}\) to \(\fun{\widetilde{f}^*}{\homs{A/H'}{X}}{\homs{A/H}{X}}\).

  Now we want to prove the naturality of the following square.
  \begin{center}
    \begin{tikzcd}[sep=huge]
      \homs{A/H}{X} \ar[r, "\psi_{EX}"] & \tgset(E, \homs{A}{X}_G) \\
      \homs{A/H'}{X} \ar[r, "\psi_{FX}"] \ar[u, "\widetilde{f}^*"] & \gset(F, \homs{A}{X}_G) \ar[u, "f^*"]
    \end{tikzcd}
  \end{center}
  As before we chase a \(x\in\homs{A/H'}{X}\) down the two possible paths to get two arrows of type \(E\to\homs{A}{X}_G\) and prove that they are the same by evaluating them on \(e\in E\):
  \[(\psi_{EX}\circ \widetilde{f}^*)(x)(e) = \psi_{EX}(x\circ \widetilde{f})(e) = x\circ \widetilde{f}\circ q = x\circ q',\]
  \[(f^*\circ \psi_{FX})(x)(e) = (\psi_{FX}(x)\circ f)(e) = \psi_{FX}(x)(f(e)) = \psi_{FX}(x)(e) = x\circ q'.\]
  Note that we used that \(f(e) = e\), this holds because the action of \(G\) on \(G/H \cong E\) and \(G/H'\cong F\) is just left multiplication and \(f\) is \(G\)-invariant.
\end{proof}

\end{document}

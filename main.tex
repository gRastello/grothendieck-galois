\documentclass[italian, 12pt, reqno]{article}
\usepackage[english]{babel}
\usepackage[T1]{fontenc}

% Stylistic points.
\usepackage{geometry}
\geometry{
  a4paper,
  right=2.5cm,
  left=2.5cm,
  top=2.5cm,
  bottom=2.5cm,
  %showframe,
}

\usepackage{mathpazo} % Math Palatino font.

% Links and references.
\usepackage{xcolor}
\definecolor{Myblue}{rgb}{0,0,0.6}
\usepackage[a4paper,colorlinks,citecolor=Myblue,linkcolor=Myblue,urlcolor=Myblue,pdfpagemode=None]{hyperref}

% Necessities for math.
\usepackage{amsmath, amscd, amssymb, mathrsfs, accents, amsfonts, amsthm}

\newtheoremstyle{myteo}{\topsep}{\topsep}
	{}
	{}
	{\bfseries}
        {.}
	{2pt}
	{\thmname{#1}\thmnumber{ #2}\thmnote{ (#3)}}
\theoremstyle{myteo}

\newtheorem{theorem}{Theorem}[section]
\newtheorem{proposition}[theorem]{Proposition}
\newtheorem{lemma}[theorem]{Lemma}
\newtheorem{corollary}[theorem]{Corollary}
\newtheorem{definition}[theorem]{Definition}
\newtheorem{example}[theorem]{Example}
\newtheorem{remark}[theorem]{Remark}

\numberwithin{equation}{section}

\usepackage{tikz}
\usetikzlibrary{cd}
\usetikzlibrary{fadings}

% Figures stuff.
\usepackage{caption}
\renewcommand{\thefigure}{\arabic{section}.\arabic{figure}}

% Lists stuff.
\usepackage{enumitem}
\setenumerate{label=(\arabic*)}

% Commands.
\newcommand{\cat}[1]{\mathscr{#1}}
\newcommand{\dual}[1]{#1^{\text{op}}}
\newcommand{\fun}[3]{#1\colon #2\to #3}

% Operators
\DeclareMathOperator{\im}{im}

\begin{document}
\title{Grothendieck's Galois Theory}
\author{Gabriele Rastello}
\maketitle

\section{Section 1}
\label{sec:1}

\begin{definition}
  \label{def:strit_epi}
  In a category \(\cat{C}\) an arrow \(\fun{f}{X}{Y}\) is a \textbf{strict epimorphism} if it is the joint coequalizer of all the arrows it equalizes.
  This means that any arrow \(\fun{g}{X}{Z}\) such that \(g\circ x = g\circ y\) for any \(\fun{x,y}{C}{X}\) such that \(f\circ x = f\circ y\) there exists a unique arrow \(\fun{h}{Y}{Z}\) such that \(h\circ f = g\).
  Refer to Figure \ref{diagram:strict_epi}.
\end{definition}

\begin{figure}[h]
  \begin{center}
    \begin{tikzcd}[sep=huge]
      C \ar[r, shift right=1, "x"'] \ar[r, shift left=1, "y"] & X \ar[r, "f"] \ar[dr, "g"] & Y \ar[d, dashed, "h"]\\
      & & Z
    \end{tikzcd}
  \end{center}
  \caption{}
  \label{diagram:strict_epi}
\end{figure}

\begin{remark}
  \label{rem:strict_epi}
  Strict epimorphisms are coequalizers, thus epimorphisms (as the name implies).
\end{remark}

\begin{remark}
  \label{rem:strict_epi_plus_mono}
  If an arrow is both a stric epimorphism and a monomorphism then it is an epimorphism.
\end{remark}

\end{document}
